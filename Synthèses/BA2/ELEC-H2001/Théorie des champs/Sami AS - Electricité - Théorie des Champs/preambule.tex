

%% Packages élémentaires %%
\usepackage[utf8]{inputenc}
\usepackage[french]{babel}

% Police
\usepackage{mathpazo}

%
\usepackage{etoolbox}

% Images, mise en page
\usepackage{graphicx, wrapfig, pbox, fancybox, hyperref, appendix, longtable, geometry}
\graphicspath{{images/}}
\geometry{hmargin=2.4cm, vmargin = 2.1cm}

% Maths
\usepackage{amsmath}
\usepackage{wasysym}

% Enumérations
\usepackage{calc}  
\usepackage{enumitem}

%% Couleurs %%
\usepackage[dvipsnames]{xcolor}
\definecolor{bleu}{RGB}{14, 68, 175}
\definecolor{bleu3}{RGB}{222, 233, 255 }
\definecolor{orange2}{RGB}{255, 216, 154}
\definecolor{rouge}{RGB}{201, 0, 0}
\definecolor{vert}{RGB}{14, 137, 0}
\newcommand\rouge[1] {{\color{rouge}{#1}}}
\newcommand\bleu[1] {{\color{bleu}{#1}}}
\newcommand\green[1]{{\color{vert}{#1}}}

%% Cadres %%
\newcommand\bb[1]{
\begin{center}
\fcolorbox{black}{bleu3}{\parbox{\textwidth}{ \begin{center}
\begin{Large}
#1
\end{Large}
\end{center}}}
\end{center}}

\newcommand\bo[1]{
\begin{center}
\fcolorbox{black}{orange2}{\parbox{\textwidth}{ \begin{center}
\begin{Large}
#1
\end{Large}
\end{center}}}
\end{center}}

\newcommand\boite[1]{
\begin{center}
\fbox{\parbox{\textwidth}{ \begin{center}
\begin{Large}
#1
\end{Large}
\end{center}}}
\end{center}}

%% Commandes %%
\newcommand\imp[1]{\underline{\textbf{#1}}}
\newcommand\eq[1]{\begin{Large}
\begin{align*}#1\end{align*}\end{Large}}
\newcommand\equ[2]{\begin{Large}
\begin{equation} \label{#1}#2\end{equation}\end{Large}}
\newcommand\bfr[3]{\begin{wrapfigure}{r}{#1\textwidth}
\centering
\includegraphics[scale=#2]{#3}
\end{wrapfigure}}
\newcommand \maxwelln{
\begin{eqnarray*}
\oint_S \vec{E}\cdot d\vec{S} &=& \dfrac{Q}{\varepsilon_0} \\
\oint_C \vec{E} \cdot d\vec{l} &= &-\dfrac{d}{dt} \int_S \vec{B} \cdot d\vec{S} \\
\oint_S \vec{B}\cdot d\vec{S} &= &0 \\
\oint_C \vec{B} \cdot d\vec{l} &=& \mu_0 I + \varepsilon_0  \mu_0 \dfrac{d}{dt} \int_S \vec{E} \cdot d\vec{S}
\end{eqnarray*}}
\newcommand \maxwell{
\begin{eqnarray}
\oint_S \vec{E}\cdot d\vec{S} &=& \dfrac{Q}{\varepsilon_0} \\
\oint_C \vec{E} \cdot d\vec{l} &= &-\dfrac{d}{dt} \int_S \vec{B} \cdot d\vec{S} \\
\oint_S \vec{B}\cdot d\vec{S} &= &0 \\
\oint_C \vec{B} \cdot d\vec{l} &=& \mu_0 I + \varepsilon_0 \mu_0 \dfrac{d}{dt} \int_S \vec{E} \cdot d\vec{S}
\end{eqnarray}}
\newcommand \maxwellvide{
\begin{eqnarray*}
\oint_S \vec{E}\cdot d\vec{S} &=& \dfrac{Q}{\varepsilon_0} \\
\oint_C \vec{E} \cdot d\vec{l} &= &0 \\
\oint_S \vec{B}\cdot d\vec{S} &= &0 \\
\oint_C \vec{B} \cdot d\vec{l} &=& \mu_0 I 
\end{eqnarray*}}

\newcommand\bfe[3]{\begin{wrapfigure}[#1]{r}{#2\textwidth}
\centering
#3
\end{wrapfigure}}

\newcommand\wraparray[3]{\begin{wrapfigure}[#1]{r}{#2\textwidth}
\centering
\vspace{-1cm}
\begin{eqnarray*}
#3
\end{eqnarray*}
\end{wrapfigure}}
\newcommand\Rn{\mathbb{R}^n}
\renewcommand\div[1]{\text{div}#1}
\newcommand\rot[1]{\overrightarrow{\text{rot}} #1}
\newcommand\grad[1]{\overrightarrow{\text{grad}} #1}

%% Commandes fantaisistes (cf. Internet) %%
\renewcommand{\parallel}{ \mathbin{\!/\mkern-5mu/\!} }
\newcommand{\q}[1]{{%
\font\larm = larm1000%
\larm%
\char 190}{\textit{ #1 }}{%
\font\larm = larm1000%
\larm%
\char 191}}
